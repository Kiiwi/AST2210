When light from a point object that is not on the optical axis is observed, we
may define two planes. The plane spanned by the optical axis and the chief ray
(the line from the point object through the center of the lens) is the
tangential plane. The sagittal plane is the plane that contains the chief ray
and is orthogonal to the tangential lane. Rays in the sagittal plane has a
shorter focal length than the tangential rays, so they will not come into focus
at the same position. If all light rays are in these planes, the image will form
a cross, or a line, if the image is viewed in one of the focal point.

To experience astigmatic aberration we send a laser beam through a lense
dsigned for enhancing astigmatism. The focal length is $100 \U{mm}$ and this is
roughly were the secondary lense was placed. Again, the black and white camera
was used.

\bilde{ast2210_lab2_pics/astigmatism/astigmatism_09.0mm.png}{fig:astigmatism1}
      {1.0 mm inside best focus. ($303 \times 303$ pixels)}
\bilde{ast2210_lab2_pics/astigmatism/astigmatism_10.0mm.png}{fig:astigmatism2}
      {Best focus. ($322 \times 266$ pixels)}
\bilde{ast2210_lab2_pics/astigmatism/astigmatism_10.5mm.png}{fig:astigmatism3}
      {0.5 mm outside best focus. ($458 \times 405$ pixels)}

Examples of three recorded astigmatism crosses are shown in figures
\ref{fig:astigmatism1} to \vref{fig:astigmatism3}. Notice that primary
distortion in figure \ref{fig:astigmatism1} and \ref{fig:astigmatism3} are
orthogonal to eachother. These are roughly the sagittal and tangential foci,
respectively. By ``best focus'' we mean the position where the average
distortion semed to be the smallest. Note that these images have different zooms
to highlight the interesting part. The total image size is $752 \times 480$
pixels. In reality, not all rays are in one of the mentioned planes, so the
resulting structure is more complex than a simple cross.
