\bilde{ast2210_lab2_pics/coma/coma.png}{fig:coma}{Source: Wikipedia}

For this part, we set the laser to send its light through
a lens on a mount with adjustable tilt, and to the black-and-white camera
linked to the computer. This lens is intentionally made to have a strong coma
and has a focal length of $150 \U{mm}$, and this is roughly where we placed the
secondary lens. Figure \vref{fig:coma_setup} describes the setup. Later in the
experiment, we added a circular aperture between the lamp and the lens.

\bilde{ast2210_lab2_pics/equipment/IMG_0468.JPG}{fig:coma_setup}
    {Setup for the experiment on coma. Black-and-white camera, microscope
    objective, lens with strong coma. A tube with a laser source is outside the
    image to the right. This setup is also used for the experiment on
    astigmatism, but with another primary lens.}

Coma, or comatic aberration, refers to when imperfect lenses 
(or other components) cause the image projection of an off-axis 
light source to be distorted in a way that makes the light source
appear to have a trail of blurry circles directed away from the 
optical axis. For this experiment, we used a lens specificially 
designed to give coma. 

In the figure \vref{fig:coma}, we see the idea behind coma, 
simply that the incoming light rays doesn't get focused to the same point.


\bilde{ast2210_lab2_pics/coma/coma_0.75mm.png}{fig:coma1}{7.5 mm offset}
\bilde{ast2210_lab2_pics/coma/coma_0.63mm.png}{fig:coma2}{6.3 mm offset}
\bilde{ast2210_lab2_pics/coma/coma_0.50mm.png}{fig:coma3}{5.0 mm offset}

Looking at figures \vref{fig:coma1}-\vref{fig:coma3}, we clearly see
the aforementioned trail, resulting in a very blurry and unfocused light dot.

We now introduce a circular aperture, placed between the light source and the
lens.

In figure \vref{fig:ap1}, the aperture was set to 19 mm, and we already see
a much clearer image. Coma is still very present, but reducing the aperture
further to 10 mm, and we have a nice and clear dot. The trails are heavily 
reduced, if not gone, due to the aperture filtering out the light rays that 
are "most" off-axis, so that we collect the ones that are the most focused.

\bilde{ast2210_lab2_pics/coma/coma_0.63mm._aperture19mm.png}{fig:ap1}{6.3 mm offset, 19 mm aperture}
\bilde{ast2210_lab2_pics/coma/coma_0.63mm._aperture10mm.png}{fig:ap2}{6.3 mm offset, 10 mm aperture}
