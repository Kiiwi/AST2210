\bilde{ast2210_lab2_pics/spherical/inside-extreme.png}
      {fig:spherical1}{Far inside best focus.}
\bilde{ast2210_lab2_pics/spherical/inside.png}{fig:spherical2}
      {Slightly inside best focus.}
\bilde{ast2210_lab2_pics/spherical/best.png}{fig:spherical3}
      {Best focus.}
\bilde{ast2210_lab2_pics/spherical/outside.png}{fig:spherical4}
      {Slightly outside best focus.}
\bilde{ast2210_lab2_pics/spherical/outside-extreme.png}{fig:spherical5}
      {Far outside best focus.}

For any spherical lens or mirror, the focal point will vary to some degree with 
the distance from the lens center where a light ray strikes, because they are 
bent too much near the edges compared to near the center. There is no point 
where all the rays meet to create a clear image. The best we can do is to find 
the point of where the disk of confusion is the smallest, i.e. where the 
average distance to the optical axis for the light rays are as small as 
possible.

While experimenting with a point light source through a lens with a significant 
spherical aberration, the best focus we could produce is shown in figure 
\vref{fig:spherical}. If the image is viewed inside this best focus the high 
refraction on the edges of the lens causes the light rays to compress along the 
rim on the blurred image, giving it sharp edges, as seen in figues 
\vref{fig:spherical1} and \vref{fig:spherical2}. In a manner we can achieve a 
\emph{circle} where the light is focused around a blurry disk, and this seems 
to happen in figure \ref{fig:spherical1}.