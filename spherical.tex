For any spherical lens or mirror, the focal point will vary to some degree with 
the distance from the lens center where a light ray strikes, because they are 
bent too much near the edges compared to near the center. There is no point 
where all the rays meet to create a clear image. The best we can do is to find 
the point where the disk of confusion is the smallest, i.e. where the
average distance from the optical axis to the light rays are as small as
possible.

The same setup is used as in the chromatic aberration section, since the
singlet lens have a significant spherical aberration. The white light source is
replaced with a laser. Instead of recording the image directly we placed the
cover on the camera and simply project the image on the cover. We used a mobile
phone camera to take pictures of this projection.

\bilde{ast2210_lab2_pics/spherical/inside-extreme.png}
      {fig:spherical1}{Far inside best focus.}
\bilde{ast2210_lab2_pics/spherical/inside.png}{fig:spherical2}
      {Slightly inside best focus.}
\bilde{ast2210_lab2_pics/spherical/best.png}{fig:spherical3}
      {Best focus.}
\bilde{ast2210_lab2_pics/spherical/outside.png}{fig:spherical4}
      {Slightly outside best focus.}
\bilde{ast2210_lab2_pics/spherical/outside-extreme.png}{fig:spherical5}
      {Far outside best focus.}

By shifting the position of the secondary lens back and forth a few millimetres
the focus point of the laser light is shifted accordingly. The best focus we
could produce
is shown in figure \vref{fig:spherical3}. If the image is viewed inside this
best focus the high refraction on the edges of the lens causes the light rays
to compress along the rim on the blurred image, giving it sharp edges, as seen
in figues \vref{fig:spherical1} and \vref{fig:spherical2}. In a manner we can
achieve a \emph{circle} where the light is focused around a blurry disk, and
this seems to happen in figure \ref{fig:spherical1}. By moving the secondary
lens away from the camera (projection screen) the image is viewed outside of
its best focus. The rays that passed through the lens near the center
are more focused now, but the rays from further out have already drifted apart
and are very blurry. This can be seen in figures \ref{fig:spherical4} and
\vref{fig:spherical5}. The effect is now the opposite as when the image was
viewed inside the best focus.

Ideally, lenses and mirrors should be paraboloids, and not spherical. But
paraboloid surfaces are more dificult to shape, and therefore cheap spherical
surfaces are often used instead. If a small area of the sphere, relative to its
radius, is used, it resembles a paraboloid more closely. This means having a
long focal length. If the focal length is short the aberration becomes more
significant. Often, several spherical lenses are placed in conjuntion to
correct for spherical aberration.
