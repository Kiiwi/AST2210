Chromatic aberration is a distortion that makes light of different wavelengths not converge at the same point. The reason this happens is because the index of refraction is dependent on wavelength. Figure \vref{fig:chromatic_fig} shows what occurs when light passes the lens.

\bilde{ast2210_lab2_pics/chromatic/chromatic_fig.png}{fig:chromatic_fig}
    {Drawing to illustrate chromatic aberration.
    Shorter wavelengths have a shorter focal length.}

 Change in the index of refraction is called dispersion. This means that the dispersion properties of the lens determine the amount of chromatic aberration. To produce the aberration, the thin singlet lens was placed in front of the microscope objective and colour camera. The lens had a focal length of about 30 cm. The microscope objective was placed roughly at this point. Figure \vref{fig:oppsett1} shows the set-up. 

\bilde{ast2210_lab2_pics/equipment/IMG_0453.png}{fig:oppsett1}
    {Set-up for the experiments on chromatic and spherical aberration. Note
    the four instruments that are aligned from left to right: Tube with
    light source, singlet lens, secondary lens (microscope objective), colour
    camera.}

We used the laser to align everything before changing to the white light lamp. Some small adjustments were needed to see the image on the computer screen. We then proceeded to adjust the microscope objective along the axis of the incoming light to find the point of convergence for the different colours. We measured the distance the microscope objective was adjusted from the point of the slide that was farthest from the camera at each maximum intensity for the colours red, green and blue. The measurements were taken with a simple ruler. We estimated that the margin for error for these measurements were $\pm 0.3$ mm. We found blue at its peak intensity at 0.5 mm, green at 3.0 mm and red at 4.5 mm. The distances between red and blue is then $4.0 \pm 0.6$ mm.

\bilde{ast2210_lab2_pics/chromatic/foci/foci-blue_0.5mm.png}{fig:Blue}{Blue at peak intensity}
\bilde{ast2210_lab2_pics/chromatic/foci/foci-green_3.0mm.png}{fig:Green}{Green at peak intensity}
\bilde{ast2210_lab2_pics/chromatic/foci/foci-red_4.5.png}{fig:Red}{Red at peak intensity}

The figures \vref{fig:Blue}-\vref{fig:Red} show the each colour at their respective focus.

An Airy disk is a pattern than is caused by diffraction of the light when passing through a circular aperture. The pattern presents itself as a bright region in the centre with alternating dark and light circular regions that decrease in intensity farther away from the centre. An Airy pattern can barely be seen on the images. It is more evident around blue than red. This could be because the blue light had higher peak intensity than red. We then placed different colour filters in the white light lamp as seen in figure \vref{fig:filter}.

\bilde{ast2210_lab2_pics/equipment/IMG_0461.png}{fig:filter}
    {Colour filter inserted into the white light lamp.}

\bilde{ast2210_lab2_pics/chromatic/filter/blue-light_no-filter.png}
    {fig:filter_blue}{Blue without filter.}
\bilde{ast2210_lab2_pics/chromatic/filter/blue-light_green-filter.png}
    {fig:filter_green}{Blue light with green filter.}
\bilde{ast2210_lab2_pics/chromatic/filter/graph.png}
    {fig:filter_green}{Graph showing the intensities of the different colours with blue light and green filter.}


We will now show what happened when we used the green filter. Figure \vref{fig:filter_blue} shows blue light without the filter, while \vref{fig:filter_green} shows the blue light with the green filter.

We can see that the light has shifted towards green. The filter mostly only allows green light through, so the contribution from blue is now significantly lower than before. The Airy pattern is also now more evident. This could be because there are fewer wavelengths that will interfere with the filter on, and this makes the Airy pattern clearer since there are less excessive interference that makes the pattern seem faded. With a monochromatic laser we would expect no such fading.
