Chromatic aberration is a distortion that makes light of different wavelengths not converge at the same point. The reason this happens is because the index of refraction is dependent on wavelength. Figure \vref{fig:chromatic_fig} shows what occurs when light passes the lens.

\bilde{ast2210_lab2_pics/chromatic/chromatic_fig.png}{chromatic_fig}{Drawing to illustrate chromatic aberration}

 Change in the index of refraction is called dispersion. This means that the dispersion properties of the lens determine the amount of chromatic aberration. To produce the aberration, the thin singlet lens was placed in front of the microscope and colour camera. The lens had a focal length of about 30 cm. The microscope objective was placed roughly at this point. Figure \vref{fig:oppsett1} shows the set-up. 

\bilde{ast2210_lab2_pics/equipment/IMG_0453.jpg}{fig:oppsett1}{Set-up for this experiment}

We used the laser to align everything before changing to the white light lamp. Some small adjustments were needed to see the image on the computer screen. We then proceeded to adjust the microscope objective along the axis of the incoming light to find the point of convergence for the different colours. We measured the distance the microscope objective was adjusted from the point of the slide that was farthest from the camera at each maximum intensity for the colours red, green and blue. The measurements were taken with a simple ruler. We estimated that the margin for error for these measurements were $\pm 0.3$ mm. We found blue at its peak intensity at 0.5 mm, green at 3.0 mm and red at 4.5 mm. The distances between red and blue is then $4.0 \pm 0.6$ mm.

\bilde{ast2210_lab2_pics/chromatic/foci/foci-blue_0.5mm.png}{fig:Blue}{Blue peak intensity}
\bilde{ast2210_lab2_pics/chromatic/foci/foci-green_3.0mm.png}{fig:Green}{Green peak intensity}
\bilde{ast2210_lab2_pics/chromatic/foci/foci-red_4.5mm.png}{fig:Red}{Red peak intensity}

The figures \vref{fig:Blue}-\vref{fig:red} show the each colour at their respective focus.

