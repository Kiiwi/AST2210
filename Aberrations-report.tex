%% LyX 2.0.0 created this file.  For more info, see http://www.lyx.org/.
%% Do not edit unless you really know what you are doing.
\documentclass[twocolumn]{article}
\usepackage[T1]{fontenc}
\usepackage[utf8]{inputenc}
\setcounter{secnumdepth}{-1}
\setlength{\parskip}{\smallskipamount}
\setlength{\parindent}{0pt}
\usepackage{float}
\usepackage{amsmath}
\usepackage[english]{babel} % change to english if english
\usepackage{graphicx, color}
\usepackage{amsmath}
\usepackage{varioref} % fancy captions
%\usepackage[margin=2cm]{geometry} % smaller margins
\usepackage{grffile} % Ex­tended file name sup­port for graph­ics, allows periods in filenames

\makeatletter
%%%%%%%%%%%%%%%%%%%%%%%%%%%%%% Textclass specific LaTeX commands.
%\numberwithin{equation}{section}
%\numberwithin{figure}{section}

\makeatother

% This makes it very simple to include images:
% Example of usage: \bilde{images/coma.bmp}{fig:coma}{This is an image of coma distortion.}
\newcommand{\bilde}[3]{
    \begin{figure}[htbp]
        \centering
        \includegraphics[width=0.5\textwidth]{#1}
        \caption{#3 \label{#2}}
    \end{figure}
}
\newcommand{\bildeto}[4]{
    \begin{figure}[htbp]
        \centering
        \includegraphics[width=0.5\textwidth]{#1}
        \includegraphics[width=0.5\textwidth]{#2}
        \caption{#4 \label{#3}}
    \end{figure}
}


% Custom commands:
%\newcommand{\nameOfCommand}[numberOfArguments]{command}
\newcommand{\D}[1]{\ \mathrm{d}#1} % steps in integrals, ex: 4x \D{x} -> 4x dx
\newcommand{\E}[1]{\cdot 10^{#1}}  % exponents, ex: 1.4\E{3} -> 1.4*10^3
\newcommand{\U}[1]{\, \textrm{#1}} % display units prettily, ex: 15.4\U{m} -> 15.4 m
\newcommand{\degree}{\, ^\circ}    % make a degree symbol


\title{AST2210 - Lab exercise: Aberrations}
\author{Viljar Drevland Hardersen, Tommy Lee, Paul Magnus Sørensen-Clark}

\begin{document}
\maketitle


\section{Introduction}

In this exercise we will explore the five optical abberations that
occur when light travels through an optical system. These abberations
are chromatic, spherical, coma, astigmatism and barrel/pincushion
distortion.


\section{Instruments}

The abberations can be divided into monochromatic and chromatic abberations,
so had to use two different cameras: The Edmund Optics USB camera
and the monochromatic Edmund Optics USB camera. The colour camera
was used in conjunction with a white light lamp, while the black and
white camera was used with a laser. 


\section{Chromatic aberration}



\section{Spherical aberration}
\bilde{ast2210_lab2_pics/spherical/inside-extreme.png}{fig:}{Far inside best focus.}
\bilde{ast2210_lab2_pics/spherical/inside.png}{fig:}{Slightly inside best focus.}
\bilde{ast2210_lab2_pics/spherical/best.png}{fig:}{Bestfocus.}
\bilde{ast2210_lab2_pics/spherical/outside.png}{fig:}{Slightly outside best focus.}
\bilde{ast2210_lab2_pics/spherical/outside-extreme.png}{fig:}{Far outside best focus.}



\section{Coma}
\bilde{ast2210_lab2_pics/coma/coma.png}{fig:coma}{Source: Wikipedia}

For this part, we set a white light lamp to send it's light through
a lens on a mount with adjustable tilt, and to a monochromatic camera
linked to the computer. Later in the experiment, we added a circular
aperture between the lamp and the lens. 

Coma, or comatic aberration, refers to when imperfect lenses 
(or other components) cause the image projection of an off-axis 
light source to be distorted in a way that makes the light source
appear to have a trail of blurry circles directed away from the 
optical axis. For this experiment, we used a lens specificially 
designed to give coma. 

In the figure \vref{fig:coma}, we see the idea behind coma, 
simply that the incoming light rays doesn't get focused to the same point.


\bilde{ast2210_lab2_pics/coma/coma_0.75mm.png}{fig:coma1}{}
\bilde{ast2210_lab2_pics/coma/coma_0.63mm.png}{fig:coma2}{}
\bilde{ast2210_lab2_pics/coma/coma_0.50mm.png}{fig:coma3}{}

Looking at figures \vref{fig:coma1}-\vref{fig:coma3}, we clearly see the aforementioned
trail, resulting in a very blurry and unfocused light dot.

We now introduce a circular aperture, put between the light source and the lens.

In figure \vref{fig:ap1}, the aperture was set to 19 mm, and we already see
a much clearer image. Coma is still very present, but reducing the aperture
further to 10 mm, and we have a nice and clear dot. The trails are heavily 
reduced, if not gone, due to the aperture filtering out the light rays that 
are "most" off-axis, so that we collect the ones that are the most focused.

\bilde{ast2210_lab2_pics/coma/coma_0.63mm._aperture19mm.png}{fig:ap1}{}
\bilde{ast2210_lab2_pics/coma/coma_0.63mm._aperture10mm.png}{fig:ap2}{}


\section{Astigmatism}
\bilde{ast2210_lab2_pics/astigmatism/astigmatism_09.0mm.png}{fig:astigmatism1}
      {1.0 mm inside best focus. ($303 \times 303$ pixels)}
\bilde{ast2210_lab2_pics/astigmatism/astigmatism_10.0mm.png}{fig:astigmatism2}
      {Best focus. ($322 \times 266$ pixels)}
\bilde{ast2210_lab2_pics/astigmatism/astigmatism_10.5mm.png}{fig:astigmatism3}
      {0.5 mm outside best focus. ($458 \times 405$ pixels)}

When light from a point object that is not on the optical axis is observed, we
may define two planes. The plane spanned by the optical axis and the chief ray
(the line from the point object through the center of the lens) is the
tangential plane. The sagittal plane is the plane that contains the chief ray
and is orthogonal to the tangential lane. Rays in the sagittal plane has a
shorter focal length than the tangential rays, so they will not come into focus
at the same position. If all light rays are in these planes, the image will form
a cross, or a line, if the image is viewed in one of the focal point.

To experience astigmatic aberration we send a laser beam through a lense
dsigned for enhancing astigmatism. The focal length is $100 \{mm}$ and this is
roughly were the secondary lense was placed. Again, the black and white camera
was used.

Examples of three recorded astigmatism crosses are shown in figures
\ref{fig:astigmatism1} to \vref{fig:astigmatism3}. Notice that primary
distortion in figure \ref{fig:astigmatism1} and \ref{fig:astigmatism3} are
orthogonal to eachother. These are roughly the sagittal and tangential foci,
respectively. By ``best focus'' we mean the position where the average
distortion semed to be the smallest. Note that these images have different zooms
to highlight the interesting part. The total image size is $752 \times 480$
pixels. In reality, not all rays are in one of the mentioned planes, so the
resulting structure is more complex than a simple cross.



\section{Barrel/pincushion distortion}
\bilde{ast2210_lab2_pics/distortion/barrel_235mm_12mm.png}{fig:barrel}{Barrel distortion, the checkerboard pattern is 235 mm in front of the lens, each square has sides 1.2 mm}

For this part, we set a lens with intentional barrel distortion 
attached to a monochromatic camera, and had a checkerboard 
patterned sheet 235 mm in front of the lens, resulting in figure /vref{fig:barrel}
We see the "lines" around the center bow outwards 

Distortion can be very irregular, but follow a few patterns, mainly
pincushion and barrel distortion. Pincushion distortion happens when 
image magnification is incresed radially away from the optical axis. 
Barrel distortion is the opposite: Magnification decreases radially
from the optical axis. 


\end{document}
