%% LyX 2.0.0 created this file.  For more info, see http://www.lyx.org/.
%% Do not edit unless you really know what you are doing.
\documentclass[twocolumn]{article}
\usepackage[T1]{fontenc}
\usepackage[utf8]{inputenc}
\setcounter{secnumdepth}{-1}
\setlength{\parskip}{\smallskipamount}
\setlength{\parindent}{0pt}
\usepackage{float}
\usepackage{amsmath}
\usepackage[english]{babel} % change to english if english
\usepackage{graphicx, color}
\usepackage{amsmath}
\usepackage{varioref} % fancy captions
%\usepackage[margin=2cm]{geometry} % smaller margins
\usepackage{grffile} % Ex­tended file name sup­port for graph­ics, allows periods in filenames

\makeatletter
%%%%%%%%%%%%%%%%%%%%%%%%%%%%%% Textclass specific LaTeX commands.
%\numberwithin{equation}{section}
%\numberwithin{figure}{section}

\makeatother

% This makes it very simple to include images:
% Example of usage: \bilde{images/coma.bmp}{fig:coma}{This is an image of coma distortion.}
\newcommand{\bilde}[3]{
    \begin{figure}[htbp]
        \centering
        \includegraphics[width=0.5\textwidth]{#1}
        \caption{#3 \label{#2}}
    \end{figure}
}
\newcommand{\bildeto}[4]{
    \begin{figure}[htbp]
        \centering
        \includegraphics[width=0.5\textwidth]{#1}
        \includegraphics[width=0.5\textwidth]{#2}
        \caption{#4 \label{#3}}
    \end{figure}
}


% Custom commands:
%\newcommand{\nameOfCommand}[numberOfArguments]{command}
\newcommand{\D}[1]{\ \mathrm{d}#1} % steps in integrals, ex: 4x \D{x} -> 4x dx
\newcommand{\E}[1]{\cdot 10^{#1}}  % exponents, ex: 1.4\E{3} -> 1.4*10^3
\newcommand{\U}[1]{\, \textrm{#1}} % display units prettily, ex: 15.4\U{m} -> 15.4 m
\newcommand{\degree}{\, ^\circ}    % make a degree symbol


\title{AST2210 - Lab exercise: Aberrations}
\author{Viljar Drevland Hardersen, Tommy Lee Ryan, Paul Magnus Sørensen-Clark}

\begin{document}
\maketitle


\section{Introduction}

In this lab exercise we will explore the five of the most common optical
aberrations that occur when light travels through an optical system. These
aberrations are chromatic aberration, and the Seidel aberrations spherical,
coma, astigmatism and barrel/pincushion distortion.

We will produce each of these aberrations intentionally, display how they
appear, and discuss why they appear as they do.


\section{Instruments}

The aberrations we have examined can be divided into mono-chromatic and
chromatic aberrations, therefore
we used two different cameras: An Edmund Optics USB camera
and a mono-chromatic Edmund Optics USB camera.
The colour camera was used in conjunction with a white light lamp,
while the black-and-white camera was used with a laser.
For the chromatic and spherical experiments we used a thin singlet lens with a
focal length of approximately $30 \U{cm}$.
For the coma experiment a specially designed lens with a focal length of $15
\U{cm}$ was used. To illustrate astigmatism we used a thick doublet lens with a
focal length of $10 \U{cm}$.
In combination with these three lenses we had a microscope objective as a
secondary lens, to act as the camera's objective.
For the distortion experiment a lens with a strong barrel distortion effect
which was attached directly to the camera with an adapter.



\section{Chromatic aberration}
Chromatic aberration is a distortion that makes light of different wavelengths not converge at the same point. The reason this happens is because the index of refraction is dependent on wavelength. Figure \vref{fig:chromatic_fig} shows what occurs when light passes the lens.

%\bilde{ast2210_lab2_pics/chromatic/chromatic_fig.png}{chromatic_fig}{Drawing to illustrate chromatic aberration}

 Change in the index of refraction is called dispersion. This means that the dispersion properties of the lens determine the amount of chromatic aberration. To produce the aberration, the thin singlet lens was placed in front of the microscope and colour camera. The lens had a focal length of about 30 cm. The microscope objective was placed roughly at this point. Figure \vref{fig:oppsett1} shows the set-up. 

%\bilde{ast2210_lab2_pics/equipment/IMG_0453.jpg}{fig:oppsett1}{Set-up for this experiment}

We used the laser to align everything before changing to the white light lamp. Some small adjustments were needed to see the image on the computer screen. We then proceeded to adjust the microscope objective along the axis of the incoming light to find the point of convergence for the different colours. We measured the distance the microscope objective was adjusted from the point of the slide that was farthest from the camera at each maximum intensity for the colours red, green and blue. The measurements were taken with a simple ruler. We estimated that the margin for error for these measurements were $\pm 0.3$ mm. We found blue at its peak intensity at 0.5 mm, green at 3.0 mm and red at 4.5 mm. The distances between red and blue is then $4.0 \pm 0.6$ mm.

%\bilde{ast2210_lab2_pics/chromatic/foci/foci-blue_0.5mm.png}{fig:Blue}{Blue peak intensity}
%\bilde{ast2210_lab2_pics/chromatic/foci/foci-green_3.0mm.png}{fig:Green}{Green peak intensity}
%\bilde{ast2210_lab2_pics/chromatic/foci/foci-red_4.5mm.png}{fig:Red}{Red peak intensity}

The figures \vref{fig:Blue}-\vref{fig:red} show the each colour at their respective focus.





\section{Spherical aberration}
For any spherical lens or mirror, the focal point will vary to some degree with 
the distance from the lens center where a light ray strikes, because they are 
bent too much near the edges compared to near the center. There is no point 
where all the rays meet to create a clear image. The best we can do is to find 
the point where the disk of confusion is the smallest, i.e. where the
average distance from the optical axis to the light rays are as small as
possible.

The same setup is used as in the chromatic aberration section, since the
singlet lens have a significant spherical aberration. The white light source is
replaced with a laser. Instead of recording the image directly we placed the
cover on the camera and simply project the image on the cover. We used a mobile
phone camera to take pictures of this projection.

\bilde{ast2210_lab2_pics/spherical/inside-extreme.png}
      {fig:spherical1}{Far inside best focus.}
\bilde{ast2210_lab2_pics/spherical/inside.png}{fig:spherical2}
      {Slightly inside best focus.}
\bilde{ast2210_lab2_pics/spherical/best.png}{fig:spherical3}
      {Best focus.}
\bilde{ast2210_lab2_pics/spherical/outside.png}{fig:spherical4}
      {Slightly outside best focus.}
\bilde{ast2210_lab2_pics/spherical/outside-extreme.png}{fig:spherical5}
      {Far outside best focus.}

By shifting the position of the secondary lens back and forth a few millimetres
the focus point of the laser light is shifted accordingly. The best focus we
could produce
is shown in figure \vref{fig:spherical3}. If the image is viewed inside this
best focus the high refraction on the edges of the lens causes the light rays
to compress along the rim on the blurred image, giving it sharp edges, as seen
in figues \vref{fig:spherical1} and \vref{fig:spherical2}. In a manner we can
achieve a \emph{circle} where the light is focused around a blurry disk, and
this seems to happen in figure \ref{fig:spherical1}. By moving the secondary
lens away from the camera (projection screen) the image is viewed outside of
its best focus. The rays that passed through the lens near the center
are more focused now, but the rays from further out have already drifted apart
and are very blurry. This can be seen in figures \ref{fig:spherical4} and
\vref{fig:spherical5}. The effect is now the opposite as when the image was
viewed inside the best focus.

Ideally, lenses and mirrors should be paraboloids, and not spherical. But
paraboloid surfaces are more dificult to shape, and therefore cheap spherical
surfaces are often used instead. If a small area of the sphere, relative to its
radius, is used, it resembles a paraboloid more closely. This means having a
long focal length. If the focal length is short the aberration becomes more
significant. Often, several spherical lenses are placed in conjuntion to
correct for spherical aberration.



\section{Coma}
\bilde{ast2210_lab2_pics/coma/coma.png}{fig:coma}{Source: Wikipedia}

For this part, we set the laser to send its light through
a lens on a mount with adjustable tilt, and to the black-and-white camera
linked to the computer. This lens is intentionally made to have a strong coma
and has a focal length of $150 \U{mm}$, and this is roughly where we placed the
secondary lens. Figure \vref{fig:coma_setup} describes the setup. Later in the
experiment, we added a circular aperture between the lamp and the lens.

\bilde{ast2210_lab2_pics/equipment/IMG_0468.JPG}{fig:coma_setup}
    {Setup for the experiment on coma. Black-and-white camera, microscope
    objective, lens with strong coma. A tube with a laser source is outside the
    image to the right. This setup is also used for the experiment on
    astigmatism, but with another primary lens.}

Coma, or comatic aberration, refers to when imperfect lenses 
(or other components) cause the image projection of an off-axis 
light source to be distorted in a way that makes the light source
appear to have a trail of blurry circles directed away from the 
optical axis. For this experiment, we used a lens specificially 
designed to give coma. 

In the figure \vref{fig:coma}, we see the idea behind coma, 
simply that the incoming light rays doesn't get focused to the same point.


\bilde{ast2210_lab2_pics/coma/coma_0.75mm.png}{fig:coma1}{7.5 mm offset}
\bilde{ast2210_lab2_pics/coma/coma_0.63mm.png}{fig:coma2}{6.3 mm offset}
\bilde{ast2210_lab2_pics/coma/coma_0.50mm.png}{fig:coma3}{5.0 mm offset}

Looking at figures \vref{fig:coma1}-\vref{fig:coma3}, we clearly see
the aforementioned trail, resulting in a very blurry and unfocused light dot.
The offsets are given from an arbitrary point.

We now introduce a circular aperture, placed between the light source and the
lens.

In figure \vref{fig:ap1}, the aperture widht was set to $19 \U{mm}$, and we
already see a much clearer image. Coma is still very present, but reducing the
aperture further to $10 \U{mm}$, and we have a nice and clear dot. The trails
are heavily reduced, if not gone, due to the aperture filtering out the
light rays that are most off-axis, so that we collect the ones that are the
most focused. This happens at the cost of losing some light.

\bilde{ast2210_lab2_pics/coma/coma_0.63mm._aperture19mm.png}{fig:ap1}{6.3 mm offset, 19 mm aperture}
\bilde{ast2210_lab2_pics/coma/coma_0.63mm._aperture10mm.png}{fig:ap2}{6.3 mm offset, 10 mm aperture}



\section{Astigmatism}
\bilde{ast2210_lab2_pics/astigmatism/astigmatism_09.0mm.png}{fig:astigmatism1}
      {1.0 mm inside best focus. ($303 \times 303$ pixels)}
\bilde{ast2210_lab2_pics/astigmatism/astigmatism_10.0mm.png}{fig:astigmatism2}
      {Best focus. ($322 \times 266$ pixels)}
\bilde{ast2210_lab2_pics/astigmatism/astigmatism_10.5mm.png}{fig:astigmatism3}
      {0.5 mm outside best focus. ($458 \times 405$ pixels)}

When light from a point object that is not on the optical axis is observed, we
may define two planes. The plane spanned by the optical axis and the chief ray
(the line from the point object through the center of the lens) is the
tangential plane. The sagittal plane is the plane that contains the chief ray
and is orthogonal to the tangential lane. Rays in the sagittal plane has a
shorter focal length than the tangential rays, so they will not come into focus
at the same position. If all light rays are in these planes, the image will form
a cross, or a line, if the image is viewed in one of the focal point.

To experience astigmatic aberration we send a laser beam through a lense
dsigned for enhancing astigmatism. The focal length is $100 \{mm}$ and this is
roughly were the secondary lense was placed. Again, the black and white camera
was used.

Examples of three recorded astigmatism crosses are shown in figures
\ref{fig:astigmatism1} to \vref{fig:astigmatism3}. Notice that primary
distortion in figure \ref{fig:astigmatism1} and \ref{fig:astigmatism3} are
orthogonal to eachother. These are roughly the sagittal and tangential foci,
respectively. By ``best focus'' we mean the position where the average
distortion semed to be the smallest. Note that these images have different zooms
to highlight the interesting part. The total image size is $752 \times 480$
pixels. In reality, not all rays are in one of the mentioned planes, so the
resulting structure is more complex than a simple cross.



\section{Barrel/pincushion distortion}
\input{Distortion.tex}


\section{Errors}
Distances were measure by hand with a ruler. With an estimated guess in the lab
we say that the margin of error in the offset measurements is $\pm 0.3 \U{mm}$.
The distances between our instruments is only vaguely given, or given in a
rough number of centimetres.
The analysis of our results is primarily qualitative, not quantitative.
Our discussions have been based on describing what kind of visual effects the
different kinds of aberrations created, and that is why we accepted to not make
very accurate measurements of our setup and results.


\end{document}
